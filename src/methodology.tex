%capitulo04
\label{cap:methodology}
\noindent 
The project is divided into cycles,in which a set of activities must be performed. At the beginning of each cycle, a planning meeting is held, in which supervisor together with the proponent define and prioritize the points that will be worked during the cycle, select the activities that must be implemented and the products or artefacts that will result from the work done during that cycle.

At the end of each cycle, the proponent presents the progress of the work, through a meeting or seminar, depending on the relevance of the work done for the research progress. In such meetings, besides being a supervisor and a coordinator, GCA teachers and students will be able to participate, in person and through video conference. The objective is to promote discussions, disseminate knowledge, identify possible inconsistencies and collect new ideas about the work carried out in the cycle that ends. 

As the research progresses and incremental content is generated, the proponent participates in workshops and conferences to discuss ideas and have feedback. When these research increments are complete and mature, they are published in journals for dissemination of the results.
In addition, at the end of the cycle, the proponent and her supervisor will hold an evaluation meeting to assess the timing, productivity, learning and planning of the next cycle.
Then a new one begins, repeating the process until the whole schedule presented in Section~\ref{cap:schedule} is fulfilled.

It is worth mentioning that one of the artefacts that will be generated at the end of the cycle will be the research report, which will have periodicity defined by the supervisor.
It is also important to highlight that the GCA is formed by professors of several institutions, which are listed in the Section~\ref{sec:collaboration}. Their researches belong to different fields of research within Software Engineering, with emphasis on the Enterprise Application Integration, under which this project is anchored.

It is the intention of the proponent and her advisor that part of the research is carried out in the laboratory of the TDG Seville group, at the University of Seville in the Spain, under the supervision of Professor Dr. Inmaculada Hernández Salmerón, linked to this university. This exchange will strengthen the international collaboration activities of the Postgraduate Program in Mathematical Modeling of UNIJUI with the postgraduate program of the University of Seville and it will have the duration of one year, probably between April 2018 and March 2019, according to the scheduler.

The research can be divided into five major steps:
\begin{enumerate}
\item Research Context review. 

This first step of the research consisted in the study of the research context, including EAI, runtime system, cloud computing and big data. In this phase, the study was accompanied by meetings, seminars, which solidified the knowledge gained.
\item Literature review. 

The this step of the research was the to deepening in the fields: enterprise application integration, runtime system, cloud computing, big data, mathematical modelling, optimization techniques, statistics models and multithread programming. The databases utilized in this research are IEEE Xplore, ACM Digital Library, Scorpus. In this phase, There was production of articles, which resulted in good feedbacks.
\item Technical and scientific literature review.

The this step of the research consisted of studying runtime systems from ten known integration platforms. In this step, we identification of properties that may have an impact on the performance of runtime systems and from these properties, we found out weaknesses of the runtime systems studied, from which derived the research problems that will be approached in this thesis.
\item Proposal modelling.

In this step, we will study alternatives for runtime systems: \textit{(i)} to take advantage of the parallel programming of multiprocessors; (ii) create and configure thread pool dynamically and elastically, following the demand for tasks; (iii) has a more optimized scheduling of tasks for the thread pools; \textit{(iv)} have mechanisms to detect bottlenecks in the execution of integration solutions; \textit{(v)} knows the computational complexity of the tasks being executed.

\item Development of a prototype.

In this step, we will develop a proof of concept to measure the gains of the proposed runtime system.
\item Validation of the proposal.

In this step, we will carry out statistical validation described in Section~\ref{cap:validation}.
\item Thesis writing.

The last stage contemplates the closing of the works, evaluation of the obtained results and writing the thesis. 
\end{enumerate}